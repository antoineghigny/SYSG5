\documentclass[a4paper, 12pt]{article}




\makeatletter
\title{SYSG5 : Exploitation de failles de sécurité LINUX}
\let\Title\@title
\author{Antoine Ghigny - 56359}          \let\Author\@author
\date{29/10/2022}           \let\Date\@date
\makeatother

\usepackage[utf8]{inputenc}
\usepackage[T1]{fontenc}

\usepackage{titling}

\usepackage[a4paper,
            bindingoffset=0.2in,
            left=1in,
            right=1in,
            top=1in,
            bottom=1in,
            footskip=.25in]{geometry}
            
\usepackage{blindtext}

\usepackage{graphicx,color, caption2}
\usepackage{epsfig}
\usepackage{fancyhdr}
\pagestyle{fancy}
\usepackage{listings}
\usepackage{color}
\usepackage{makeidx}

% Valeurs par défaut le lstset
\lstset{language={},%C,Assembleur, TeX, tcl, basic, cobol, fortran, logo, make, pascal, perl, prolog, {}
	literate={â}{{\^a}}1 {ê}{{\^e}}1 {î}{{\^i}}1 {ô}{{\^o}}1 {û}{{\^u}}1
		 {ä}{{\"a}}1 {ë}{{\"e}}1 {ï}{{\"i}}1 {ö}{{\"o}}1 {ü}{{\"u}}1
		 {à}{{\`a}}1 {é}{{\'e}}1 {è}{{\`e}}1 {ù}{{\`u}}1 
		 {Â}{{\^A}}1 {Ê}{{\^E}}1 {Î}{{\^I}}1 {Ô}{{\^O}}1 {Û}{{\^U}}1
		 {Ä}{{\"A}}1 {Ë}{{\"E}}1 {Ï}{{\"I}}1 {Ö}{{\"O}}1 {Ü}{{\"U}}1
		 {À}{{\`A}}1 {É}{{\'E}}1 {È}{{\`E}}1 {Ù}{{\`U}}1,
	commentstyle=\scriptsize\ttfamily\slshape, % style des commentaires
	basicstyle=\scriptsize\ttfamily, % style par défaut
	keywordstyle=\scriptsize\rmfamily\bfseries,% style des mots-clés
	backgroundcolor=\color[rgb]{.95,.95,.95}, % couleur de fond : gris clair
	framerule=0.5pt,% Taille des bords
	frame=trbl,% Style du cadre
	frameround=tttt, % Bords arrondis 
	tabsize=3, % Taille des tabulations
%	extendedchars=\true, % Incompatible avec utf8 et literate
	inputencoding=utf8,
	showspaces=false, % Ne montre pas les espaces 
	showstringspaces=false, % Ne montre pas les espaces entre ''
	xrightmargin=-1cm, % Retrait gauche 
	xleftmargin=-1cm, % Retrait droit
	escapechar=°}  % Caractère d'échappement, permet des commandes latex dans la source


\begin{document}
\maketitle 

   \tableofcontents

   \section{Dépassement de mémoire : Pwnkit}
   		\subsection{Quel est le principe de cette faille ?}
   		
		\subsection{Origine de la faille}   		
   		\begin{itemize}
			\item 	Polkit est bibliothèque sur laquelle a été découvert cette vulnérabilité. Il a été créé à la base pour permettre aux dévelopeurs de réaliser des actions qui nécessitaient des privilèges élevés sur le système. On peut le comparer à sudo qui fait essentiellement la même chose côté utilisateur.
   			\item Cette faille va s'intéresser à l'utilisation de la commande \textbf{pkexec} qui fait partie de la bibliothèque polkit. L'appel système pkexec est apparu en 2009 et inclus dans pratiquement toutes les distributions linux actuelles.
   			\item Cette faille de sécurité est présente depuis 12 ans et récemment mise en évidence par l'équipe de recherche Qualys en février 2022. \cite{qualys}
   		\end{itemize}
   		\subsection{Comment cela fonctionne ?}
   		
   		\subsection{Démonstration}
			\subsubsection{Le code C}   		
			\lstinputlisting{Code/exploit.c}
			\subsubsection{Script qui permet d'exécuter la faille}   	
			\lstinputlisting{Code/Demo}
			\subsubsection{Le script qui permet d'exploiter cette faille}   		

   \section{Modifier le mot de passe administrateur sans le connaître}
        \subsection{chroot}
        \subsection{grub}
   \section{Bombe zip}
   		\subsection{Qu'est-ce qu'une zip bomb ?}
         Une zip bomb est un fichier compressé de quelques MO qui contient énormément de données présente sous formes d'octets, ce qui va amener à saturer le disque dur. Le système va manquer de mémoire et se bloquer dans le processus
   \section{Conclusion}

	\bibliography{bibliozq}
	\bibliographystyle{plain}
\end{document}
